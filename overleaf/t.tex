\section{Agent-Based Model Design}

Our agent-based model (ABM) simulates the electric vehicle battery lifecycle with a focus on circularity pathways. The model consists of heterogeneous agent types that interact according to defined behaviors and decision-making processes.

\subsection{Model Structure}
The model implements a Mesa-based architecture with the following key components:
\begin{itemize}
    \item A base \texttt{Agent} class that extends Mesa's agent implementation with location-based functionality
    \item Four specialized agent types: EV owners, car manufacturers, recycling facilities, and battery refurbishers
    \item A \texttt{Battery} class representing the physical batteries that flow through the system
    \item A state-transition mechanism for batteries modeled as the \texttt{BatteryStatus} enum
\end{itemize}

\subsection{Agent Types and Behaviors}

\subsubsection{EV Owners}
EV owners represent consumers who purchase and use electric vehicles with batteries. Each owner has:
\begin{itemize}
    \item Income level influencing purchasing decisions
    \item Environmental consciousness affecting end-of-life battery decisions
    \item Decision-making processes for battery replacement
    \item End-of-life battery handling with realistic waiting periods
\end{itemize}

\subsubsection{Battery}
Batteries represent physical EV battery units with:
\begin{itemize}
    \item Realistic degradation models accounting for age and usage cycles
    \item Health thresholds that trigger state transitions
    \item Capacity parameters that determine usefulness for various applications
    \item Life cycle status progression from NEW → IN\_USE → END\_OF\_LIFE → COLLECTED → RECYCLED/REFURBISHED
\end{itemize}

\subsubsection{Car Manufacturers}
Car manufacturers produce new batteries and collect end-of-life batteries:
\begin{itemize}
    \item Production capacity constraints
    \item Warranty policies based on battery age and health
    \item Battery collection commitments determining willingness to accept returns
\end{itemize}

\subsubsection{Recycling Facilities}
Recycling facilities process end-of-life batteries for material recovery:
\begin{itemize}
    \item Processing capacity limitations
    \item Operational dynamics including downtime
    \item Material recovery efficiency based on battery condition
    \item Realistic processing timeframes
\end{itemize}

\subsubsection{Battery Refurbishers}
Battery refurbishers convert suitable used batteries for second-life applications:
\begin{itemize}
    \item Technical capability affecting refurbishment success rates
    \item Capacity constraints for processing
    \item Conversion to grid storage functionality
    \item Quality assessment of incoming batteries
\end{itemize}

\subsection{System Dynamics}
The model captures key circularity pathways including:
\begin{itemize}
    \item First-life usage with realistic degradation patterns
    \item End-of-life decision-making by consumers
    \item Battery collection and assessment
    \item Recycling pathways for material recovery
    \item Second-life applications through refurbishment
\end{itemize}

Battery flows are monitored throughout the system, with realistic constraints on processing capacities, waiting times, and operational variations that affect the efficiency of circularity pathways.