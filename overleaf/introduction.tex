\section{Introduction}
The current adoption of electric vehicles (EVs) in the Netherlands but also globally is very fast, which results in a shift in transportation systems. All this is due to environmental concerns and incentive policies to reduce gas emissions \cite{IEA2023}. In the Netherlands, 34. 9\% of the sales of new cars in 2024 are electric vehicles \cite{EAFO2025}, and it is aiming to reach 100\% zero-emission vehicles by 2030 \cite{RVO2022}. But this goal raised questions about resource sustainability and the waste of these lithium-ion batteries which are used for the EVs. 

Our goal is to assess how the circularity of BEV batteries can be maximized, with a focused scope on the impact of the various second-life applications, and how this can support the Netherlands' goal of achieving 100\% circularity by 2050 \cite{CircularNetherlands2016}. We aim to visualize and better understand the situation through the development of an Agent-Based Model. Our research question addresses the urgency to develop sustainable end-of-life management solutions for EV batteries as the market share of electric vehicles in the Netherlands continues to grow rapidly \cite{CBS2023Elektrisch}.

These Lithium-ion batteries bring opportunities, but also challenges. The challenge is about the critical materials used, such as lithium, nickel and cobalt. These materials have a limited supply and have a significant environmental impact during their extraction \cite{Olivetti2017}. But they also have opportunities, these materials are valuable resources that can be recovered through recycling processes. Or when they have sufficient remaining capacity, they can be used for second-life applications like grid energy storage \cite{Bobba2018}.

The Netherlands aims to reduce the primary resource consumption with 50\% by 2030, and in 2050 they aim to transform the Netherlands into a fully circular economy \cite{CircularNetherlands2016}. Due to the fast adoption of the EVs, developing effective circular strategies for EV batteries is a essential if they want to achieve these objectives. Especially because of the expected surge in end-of-live EV batteries, indicating an increase in retired batteries from 2025 \cite{Hu2017}.

In this report, we used an agent-based modeling approach to learn more about the dynamics of battery circularity. These models are well-suited for studying emergent patterns resulting from interactions of agents with heterogeneous characteristics \cite{Bonabeau2002}. The use of this model allows us to simulate the decisions the different stakeholders, such as, EV owners, manufacturers, recycling facilities, and battery refurbishers make. And with that evaluate how their interactions influence the circularity rate.

This research extends the literature on EV battery circularity by providing a perspective that uses realistic constraints and behaviors. By analyzing different scenarios, we can provide evidence-based recommendations for the policymakers and stak 

This research contributes to the growing literature on EV battery circularity by providing a dynamic systems perspective that incorporates realistic constraints and stakeholder behaviors. By developing and analyzing different scenarios, we provide evidence-based recommendations to support Dutch policymakers and industry stakeholders in maximizing battery circularity rates and reaching national circular economy targets.