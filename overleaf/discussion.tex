\section{Discussion}
\subsection{Implications for Dutch Circular Economy Policy}
Our findings have several important implications for Dutch policy development regarding EV battery circularity. The results demonstrate that a combined approach—simultaneously enhancing refurbishment capabilities, strengthening producer responsibility, and optimizing second-life eligibility criteria—can achieve circularity rates approaching 95\%.

The Dutch government's goal of 100\% circularity by 2050 appears achievable for EV batteries if supported by appropriate policies and infrastructure investments. However, our model identifies potential bottlenecks in processing capacity that could hinder progress if not addressed proactively. The projected surge in end-of-life batteries from 2025 onwards will require strategic expansion of both recycling and refurbishment facilities to maintain high circularity rates.

The current Dutch policy framework, including the Climate Agreement and Circular Economy program, provides a foundation for promoting battery circularity \cite{NetherlandsClimateAgreement2019}. However, our results suggest that more specific policies targeting battery second life and recycling infrastructure would be beneficial. This could include:

\begin{itemize}
  \item Standardized testing and certification procedures for second-life batteries
  \item Financial incentives for grid storage projects utilizing second-life batteries
  \item Mandatory recycling targets for EV battery materials
  \item Support for domestic recycling and refurbishment capacity development
\end{itemize}

\subsection{Balancing Second Life and Recycling}
One of the key insights from our model is the complementary relationship between second-life applications and recycling. Rather than competing pathways, they represent sequential stages in an optimized circular system. By calibrating second-life eligibility thresholds appropriately, batteries can provide extended utility in less demanding applications before eventually being recycled for material recovery.

The lower threshold scenario demonstrates that expanding second-life utilization can be achieved without significantly compromising long-term material recovery goals. This finding contradicts concerns that second-life applications might "lock away" critical materials that would otherwise be recycled. Instead, our model indicates that properly managed second-life programs can substantially extend resource utilization while still enabling eventual material recovery, creating a more efficient circular system overall.

\subsection{Limitations and Future Research}
While our model provides valuable insights, it has several limitations that should be acknowledged. First, it assumes constant technical parameters over the simulation period, whereas in reality, battery technology, recycling processes, and refurbishment techniques will likely improve over time. Future research could incorporate technological learning curves and innovation diffusion to capture these dynamics.

Second, the model simplifies certain economic aspects, particularly regarding the cost structures of recycling and refurbishment operations and the market value of recovered materials and second-life products. More detailed economic modeling could enhance the assessment of financial viability and identify potential market failures requiring policy intervention.

Finally, the model does not fully account for international material flows and the global nature of EV battery supply chains. For a relatively small country like the Netherlands, international coordination will be critical for achieving circular economy objectives. Future research could extend the model to incorporate cross-border material flows and policy harmonization efforts at the European level.