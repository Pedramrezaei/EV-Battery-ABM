\section{Conclusion}
This research demonstrates that maximizing the circularity of EV batteries in the Netherlands requires a multi-faceted approach addressing technical capabilities, infrastructure capacity, policy frameworks, and stakeholder incentives. Through agent-based modeling, we have identified specific strategies that can significantly enhance battery circularity rates while supporting the country's broader sustainability objectives.

Our scenario analysis reveals several key insights:

\begin{itemize}
  \item \textbf{Combined interventions yield the best results:} The combined approach scenario achieved the highest overall circularity rate (84\%), demonstrating that simultaneously enhancing technical capabilities, expanding infrastructure, strengthening producer responsibility, and optimizing eligibility criteria creates synergistic effects.
  
  \item \textbf{Technical capability is particularly impactful:} Improving refurbishment technical capability from 0.7 to 0.9 increased second-life rates by 6 percentage points, making it one of the most effective single interventions.
  
  \item \textbf{Policy instruments matter:} Increasing manufacturer recycling commitment from 0.75 to 0.95 boosted proper end-of-life management, particularly through the recycling pathway, highlighting the importance of extended producer responsibility.
  
  \item \textbf{Second-life thresholds create trade-offs:} Lowering the second-life threshold from 0.6 to 0.4 decreased overall circularity but increased battery utilization, suggesting that policymakers must balance immediate material recovery with maximizing utility.
  
  \item \textbf{Material recovery is resilient:} Even in scenarios prioritizing second-life applications, material recovery shows strong long-term resilience, indicating that a sequential circular approach (reuse before recycling) is viable.
\end{itemize}

For Dutch policymakers and industry stakeholders, these results provide evidence-based guidance for developing the infrastructure and regulations necessary to manage the growing volume of EV batteries. We recommend a balanced policy approach that promotes both high-quality refurbishment for second-life applications and efficient recycling for material recovery, with particular emphasis on technical capability development and appropriate standardization of second-life eligibility criteria.

By implementing the recommended strategies, the Netherlands can establish itself as a leader in battery circularity while making significant progress toward its goal of 100\% circularity by 2050. Future research should explore the economic dimensions of these circular pathways in more detail and investigate how international coordination might enhance the effectiveness of national circularity strategies.